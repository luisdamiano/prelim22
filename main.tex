\documentclass{snedecorbeamer}

\usepackage{natbib}
\begin{document}

% Title page -------------------------------------------------------------------
\title{Automatic Dynamical Relevance Determination}
\subtitle{A methodology for Gaussian process regression\\
  with high-dimensional functional inputs}
\author{Luis Damiano}
\institute{Iowa State University\\
  Department of Statistics}
\date{October 6, 2022}

\begin{frame}
  \titlepage{}
\end{frame}

% Welcome message --------------------------------------------------------------
\begin{frame}
  \frametitle{Welcome to my prelim exam!}
\end{frame}

\begin{frame}
  \frametitle{Welcome to my prelim exam!}
  \framesubtitle{Rule book}

  \begin{itemize}[<+(1)->]
  \item I prepared a 40 minute presentation
  \item You are \textbf{strongly encouraged} to stop me as I speak
  \item I'll start with a lit review,\\
    \textbf{don't hesitate} to ask for clarification about what my contribution is
  \item I can't cover all the small details in just 40 min,\\
    \textbf{feel free} to ask about the details
  \end{itemize}
\end{frame}

\begin{frame}
  \frametitle{Welcome to my prelim exam!}
  \framesubtitle{Road map}

  \tableofcontents
\end{frame}

% Introduction -----------------------------------------------------------------
\section{Problem of interest}
\standout{Problem of interest}

\begin{frame}
  \frametitle{Functional input Gaussian process regression}
  \framesubtitle{Subtitle}

  Functional input Gaussian process regression for computer experiment emulation
\end{frame}

\section{What we know so far}

\begin{frame}
  \frametitle{Literature review}
  \framesubtitle{Computer experiment emulation}

  \begin{itemize}
  \item Scientific phenomena investigated via complex computer models
    \begin{itemize}
    \item Often deterministic
    \item Often expensive
    \item Legacy code
    \end{itemize}
  \end{itemize}

  \blankfootnote{
    \citep{sacks1989jones1998}
  }
\end{frame}

\begin{frame}
  \frametitle{Literature review}
  \framesubtitle{Gaussian process for emulation}

  General reference books, including the topics the cover
\end{frame}

\begin{frame}
  \frametitle{Literature review}
  \framesubtitle{Gaussian process with functional inputs}

  Specific lit review
\end{frame}

\begin{frame}
  \frametitle{My role on this}
  \framesubtitle{Gaussian process with functional inputs}

  What I'm doing: I'm taking some previous ideas that have been limited applied
  and (i) I'm generalizing under the name of ``Automatic Dynamical Relevance
  Determination'', (ii) introducing a new parametric the weight function,
  (iii) fiddling with basis expansions to create a wide flexible of weight
  families

  Overall, I strive for data-driven and hands-off Gaussian process regressions
  with functional inputs.

  Something you don't have to put a lot of time on it
  Something that works a reasonable default: a reasonable modus operandi
  Something with the potential to become the ``the first thing to try'' when
  working with functional inputs
  Something that doesn't rely on ``data pre-processing'', which introduces some
disconnection between the data and the modeling framework, but that happens
within the probabilistic framework offer by a GP.
  Something that's interpretable
\end{frame}

% Chapter 1 --------------------------------------------------------------------
\section{Chapter 1: ADRD + PDFI + asymmetrical Laplace function}
\standout{Chapter 1: ADRD + PDFI + asymmetrical Laplace function}

\begin{frame}
  \frametitle{Chapter 1 is BOGO}
  \framesubtitle{}

  Chapter1 can be better though in two parts:
  Chapter0 moving from ``viGP'' to the concept of a ``fiGP''
  Chapter1 introducing the ALF as the first species in the family
\end{frame}

\begin{frame}
  \frametitle{Automatic Dynamic Relevance Determination}
  \framesubtitle{}

  From ARD to ADRD
\end{frame}

\begin{frame}
  \frametitle{Permutation Dynamic Feature Importance}
  \framesubtitle{}

  From PFI to PDFI
\end{frame}

\begin{frame}
  \frametitle{Asymmetrical Laplace function}
  \framesubtitle{}

  ALF spec, priors, figures
\end{frame}

\begin{frame}
  \frametitle{Case study}
  \framesubtitle{}

  One slide introducing the science
  One slide introducing the stats
  Two slides with the results
\end{frame}

\begin{frame}
  Dump some stuff from previous presentations

\end{frame}

\section{Chapter 2: Fourier basis functions}
\standout{Chapter 2: Fourier basis functions}

\begin{frame}
  \frametitle{Fourier expansion for the weight function}
  \framesubtitle{Fourier series review}

\end{frame}

\begin{frame}
  \frametitle{Fourier expansion for the weight function}
  \framesubtitle{FEW specification goes here}

\end{frame}

\begin{frame}
  \frametitle{Fourier expansion for the weight function}
  \framesubtitle{Alternative parametrization}

\end{frame}

\begin{frame}
  \frametitle{Fourier expansion for the weight function}
  \framesubtitle{Simulation study}

  Sim study specification goes here
  Two questions: (i) ALF and FEW vs the world
  (ii) Overfitting

\end{frame}

\begin{frame}
  \frametitle{Fourier expansion for the weight function}
  \framesubtitle{Limitations}

\end{frame}

\section{Chapter 3: better basis functions}
\standout{Chapter 3: better basis functions}

\begin{frame}

  Early results suggest that Fourier basis are nice but limited

  Concepts to explore:
  \begin{itemize}
  \item Partition of unity of a topological space $\mathcal{T}$ is a set of
    continuous functions from $\mathcal{T}\to[0, 1]$ s.t.
  \item Wavelets
  \item Bernstein polynomials and Bézier curves
  \item Nonnegative B-splines
  \end{itemize}
\end{frame}

% Lose ends --------------------------------------------------------------------
\standout{Short notes}
\section{Loose ends}

\begin{frame}
  \frametitle{IBEX mission}
  \framesubtitle{}

  Summer internship at LANL
\end{frame}

\begin{frame}
  \frametitle{Analytical solution to piece-wise linear inputs}
  \framesubtitle{}

\end{frame}

% Closing slide ----------------------------------------------------------------
\begin{frame}[c]
  \frametitle{Acknowledgments}
  \centering

  {\small Jarad Niemi, Max D. Morris,\\
    Margaret Johnson (JPL), Joaquim Texeira (JPL), \\
    Microwave Limb Sounder team (JPL),\\
    David Osthus (LANL), Brian Weaver (LANL) \\
    ISU PIRI on C-CHANGE:~Science for a Changing Agriculture.\\
    Foundation for Food and Agriculture Research}

  \vfill

  {\huge Thank you!}

  \vfill

  {\tiny References and extra slides on the back}

  \href{ldamiano@iastate.edu}{\beamergotobutton{mail}
    ldamiano@iastate.edu}

  \href{https://luisdamiano.github.io/}{\beamergotobutton{site}
    luisdamiano.github.io}

  \vfill

  {\tiny Am I candidate enough?}

\end{frame}

% Appendix ---------------------------------------------------------------------
\appendix

\section{References}

\setbeamertemplate{bibliography item}{\insertbiblabel}

\begin{frame}[allowframebreaks]{References}
  \tiny
  \bibliographystyle{unsrt}
  \bibliography{references}
\end{frame}

\end{document}

%%% Local Variables:
%%% eval: (TeX-run-style-hooks "beamer")
%%% mode: latex
%%% TeX-master: t
%%% End:
